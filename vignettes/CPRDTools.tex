% Options for packages loaded elsewhere
\PassOptionsToPackage{unicode}{hyperref}
\PassOptionsToPackage{hyphens}{url}
%
\documentclass[
]{article}
\usepackage{amsmath,amssymb}
\usepackage{lmodern}
\usepackage{iftex}
\ifPDFTeX
  \usepackage[T1]{fontenc}
  \usepackage[utf8]{inputenc}
  \usepackage{textcomp} % provide euro and other symbols
\else % if luatex or xetex
  \usepackage{unicode-math}
  \defaultfontfeatures{Scale=MatchLowercase}
  \defaultfontfeatures[\rmfamily]{Ligatures=TeX,Scale=1}
\fi
% Use upquote if available, for straight quotes in verbatim environments
\IfFileExists{upquote.sty}{\usepackage{upquote}}{}
\IfFileExists{microtype.sty}{% use microtype if available
  \usepackage[]{microtype}
  \UseMicrotypeSet[protrusion]{basicmath} % disable protrusion for tt fonts
}{}
\makeatletter
\@ifundefined{KOMAClassName}{% if non-KOMA class
  \IfFileExists{parskip.sty}{%
    \usepackage{parskip}
  }{% else
    \setlength{\parindent}{0pt}
    \setlength{\parskip}{6pt plus 2pt minus 1pt}}
}{% if KOMA class
  \KOMAoptions{parskip=half}}
\makeatother
\usepackage{xcolor}
\IfFileExists{xurl.sty}{\usepackage{xurl}}{} % add URL line breaks if available
\IfFileExists{bookmark.sty}{\usepackage{bookmark}}{\usepackage{hyperref}}
\hypersetup{
  pdftitle={CPRDTools: a CPRD data wrangling toolkit},
  hidelinks,
  pdfcreator={LaTeX via pandoc}}
\urlstyle{same} % disable monospaced font for URLs
\usepackage{color}
\usepackage{fancyvrb}
\newcommand{\VerbBar}{|}
\newcommand{\VERB}{\Verb[commandchars=\\\{\}]}
\DefineVerbatimEnvironment{Highlighting}{Verbatim}{commandchars=\\\{\}}
% Add ',fontsize=\small' for more characters per line
\usepackage{framed}
\definecolor{shadecolor}{RGB}{248,248,248}
\newenvironment{Shaded}{\begin{snugshade}}{\end{snugshade}}
\newcommand{\AlertTok}[1]{\textcolor[rgb]{0.94,0.16,0.16}{#1}}
\newcommand{\AnnotationTok}[1]{\textcolor[rgb]{0.56,0.35,0.01}{\textbf{\textit{#1}}}}
\newcommand{\AttributeTok}[1]{\textcolor[rgb]{0.77,0.63,0.00}{#1}}
\newcommand{\BaseNTok}[1]{\textcolor[rgb]{0.00,0.00,0.81}{#1}}
\newcommand{\BuiltInTok}[1]{#1}
\newcommand{\CharTok}[1]{\textcolor[rgb]{0.31,0.60,0.02}{#1}}
\newcommand{\CommentTok}[1]{\textcolor[rgb]{0.56,0.35,0.01}{\textit{#1}}}
\newcommand{\CommentVarTok}[1]{\textcolor[rgb]{0.56,0.35,0.01}{\textbf{\textit{#1}}}}
\newcommand{\ConstantTok}[1]{\textcolor[rgb]{0.00,0.00,0.00}{#1}}
\newcommand{\ControlFlowTok}[1]{\textcolor[rgb]{0.13,0.29,0.53}{\textbf{#1}}}
\newcommand{\DataTypeTok}[1]{\textcolor[rgb]{0.13,0.29,0.53}{#1}}
\newcommand{\DecValTok}[1]{\textcolor[rgb]{0.00,0.00,0.81}{#1}}
\newcommand{\DocumentationTok}[1]{\textcolor[rgb]{0.56,0.35,0.01}{\textbf{\textit{#1}}}}
\newcommand{\ErrorTok}[1]{\textcolor[rgb]{0.64,0.00,0.00}{\textbf{#1}}}
\newcommand{\ExtensionTok}[1]{#1}
\newcommand{\FloatTok}[1]{\textcolor[rgb]{0.00,0.00,0.81}{#1}}
\newcommand{\FunctionTok}[1]{\textcolor[rgb]{0.00,0.00,0.00}{#1}}
\newcommand{\ImportTok}[1]{#1}
\newcommand{\InformationTok}[1]{\textcolor[rgb]{0.56,0.35,0.01}{\textbf{\textit{#1}}}}
\newcommand{\KeywordTok}[1]{\textcolor[rgb]{0.13,0.29,0.53}{\textbf{#1}}}
\newcommand{\NormalTok}[1]{#1}
\newcommand{\OperatorTok}[1]{\textcolor[rgb]{0.81,0.36,0.00}{\textbf{#1}}}
\newcommand{\OtherTok}[1]{\textcolor[rgb]{0.56,0.35,0.01}{#1}}
\newcommand{\PreprocessorTok}[1]{\textcolor[rgb]{0.56,0.35,0.01}{\textit{#1}}}
\newcommand{\RegionMarkerTok}[1]{#1}
\newcommand{\SpecialCharTok}[1]{\textcolor[rgb]{0.00,0.00,0.00}{#1}}
\newcommand{\SpecialStringTok}[1]{\textcolor[rgb]{0.31,0.60,0.02}{#1}}
\newcommand{\StringTok}[1]{\textcolor[rgb]{0.31,0.60,0.02}{#1}}
\newcommand{\VariableTok}[1]{\textcolor[rgb]{0.00,0.00,0.00}{#1}}
\newcommand{\VerbatimStringTok}[1]{\textcolor[rgb]{0.31,0.60,0.02}{#1}}
\newcommand{\WarningTok}[1]{\textcolor[rgb]{0.56,0.35,0.01}{\textbf{\textit{#1}}}}
\usepackage{graphicx}
\makeatletter
\def\maxwidth{\ifdim\Gin@nat@width>\linewidth\linewidth\else\Gin@nat@width\fi}
\def\maxheight{\ifdim\Gin@nat@height>\textheight\textheight\else\Gin@nat@height\fi}
\makeatother
% Scale images if necessary, so that they will not overflow the page
% margins by default, and it is still possible to overwrite the defaults
% using explicit options in \includegraphics[width, height, ...]{}
\setkeys{Gin}{width=\maxwidth,height=\maxheight,keepaspectratio}
% Set default figure placement to htbp
\makeatletter
\def\fps@figure{htbp}
\makeatother
\setlength{\emergencystretch}{3em} % prevent overfull lines
\providecommand{\tightlist}{%
  \setlength{\itemsep}{0pt}\setlength{\parskip}{0pt}}
\setcounter{secnumdepth}{-\maxdimen} % remove section numbering
\newlength{\cslhangindent}
\setlength{\cslhangindent}{1.5em}
\newlength{\csllabelwidth}
\setlength{\csllabelwidth}{3em}
\newlength{\cslentryspacingunit} % times entry-spacing
\setlength{\cslentryspacingunit}{\parskip}
\newenvironment{CSLReferences}[2] % #1 hanging-ident, #2 entry spacing
 {% don't indent paragraphs
  \setlength{\parindent}{0pt}
  % turn on hanging indent if param 1 is 1
  \ifodd #1
  \let\oldpar\par
  \def\par{\hangindent=\cslhangindent\oldpar}
  \fi
  % set entry spacing
  \setlength{\parskip}{#2\cslentryspacingunit}
 }%
 {}
\usepackage{calc}
\newcommand{\CSLBlock}[1]{#1\hfill\break}
\newcommand{\CSLLeftMargin}[1]{\parbox[t]{\csllabelwidth}{#1}}
\newcommand{\CSLRightInline}[1]{\parbox[t]{\linewidth - \csllabelwidth}{#1}\break}
\newcommand{\CSLIndent}[1]{\hspace{\cslhangindent}#1}
\ifLuaTeX
  \usepackage{selnolig}  % disable illegal ligatures
\fi

\title{CPRDTools: a CPRD data wrangling toolkit}
\author{}
\date{\vspace{-2.5em}}

\begin{document}
\maketitle
\begin{abstract}
The analysis of large scale data, and moreover the analysis of large
scale electronic health records data is become more commonplace. The
ease and ability of modern data generation and capture means there is
the potential across almost all industries to capture more data now,
than ever before and that equates to large data resources, with CPRD no
exception. These large data resources, as attractive and appealing as
they may be to researchers, pose a significant problem - how to manage
all that data? CPRDTools is a collection of wrapper R functions intended
to simplify the loading, extraction and management of Clinical Practice
Research Datalink (CPRD) GOLD specific and associated electgronic health
records data. Allowing for the loading of CPRD and non-CPRD data into a
SQLite based database, providing an efficient, secure and updatable
repository for the data, keeing the origional source files intact.
Through data queries, user-defined data are drawn allowing for
subsetting, joining and filtering in a signle step, creating alaysis
ready data.
\end{abstract}

\hypertarget{introduction}{%
\section{Introduction}\label{introduction}}

The CPRD is one of the largest longitudinal medical records databases in
the world, supported by the National Institute of Health (NIHR) and the
Medicines and Healthcare products Regulatory Agency (MHRA). It was first
established in 1987 as the Value Added Medical Products (VAMP) dataset,
this grew into the General Practice Research Database (GPRD) in 1993,
before its final transition into CPRD in 2012 (Herrett et al. 2015).

PRD GOLD data are comprised of ten separate datasets: patient, practice,
staff, consultation, clinical, additional clinical details, referral,
immunisation, test and therapy (Padmanabhan 2017). These datasets
contain their specific data and are linkable through a unique linkage
key field, where key does not imply importance but a unique variable
contained in two datasets allowing them to be joined, such as the
CPRD-assigned and anonymised unique patient identifier \texttt{patid}.

Due to the size of CPRD, data extracted for research are spread over
multiple text (\texttt{.txt}) files within each dataset to enable file
transfer. This means that for CPRD clinical data, for instance, a
researcher may receive their requested clinical data broken up over 25
individual text files. This is done to aid with file completeness and
reduce turn-around times if errors are found. If an error occurred
during the transfer of data between the data owner and the researcher or
in the extraction of the requested data by the data owner, the error can
potentially be limited to only select files, requiring only their
replacement with the corrected/error-free versions.

Often these text files are additionally zipped, or compressed, requiring
that these files first be uncompressed or unzipped. These multiple files
from each dataset (clinical, referral etc.) then need to be grouped
together and amalgamated into a single \texttt{table}. Tables are a
collection of data of the same shape, from the same dataset. In CPRD,
each separate dataset (patient, practice, clinical etc.) forms a table.
These tables are then stored in the SQLite database.

SQLite is an opensource, SQL based database engine (SQLite 2017). The
use of a SQLite database provides an efficient storage solution,
allowing for the loading, updating and maintenance of the database, all
while retaining the original \emph{raw} data files unaltered. An SQlite
database permits for rapid data extraction through the use of data
queries, drawing the required data from the database, allowing
filtering, sub-setting, limiting and the joining of data in a single
execution step.

This document aims to provide a simple and introductory overview of
\texttt{CPRDTools} and its application to arbitrary (fictional) data.
This data though are provided in the manner in which many CPRD data
extract are received, where data are spread over multiple files and
often located in sub-folders.

\texttt{CPRDTools} are loaded using:

\begin{Shaded}
\begin{Highlighting}[]
\FunctionTok{library}\NormalTok{(devtools)}
\FunctionTok{install\_github}\NormalTok{(}\StringTok{"JamesCFSchmidt/CPRDTools"}\NormalTok{)}
\FunctionTok{library}\NormalTok{(CPRDTools)}
\end{Highlighting}
\end{Shaded}

\hypertarget{cprdtools-overview}{%
\section{CPRDTools overview}\label{cprdtools-overview}}

The functions within `CPRDTools' broadly fall into three groups,
categorised by their general application area: (1) - loading, (2) -
maintenance and other tasks and (3) - extraction. \textbf{Loading}
encompasses all functions used in the reading, converting and writing of
data into the database including a function used to list all available
files and all available CPRD files in a specified location, functions
used in database maintenance, query speed improvements and date
conversion functions fall under \textbf{maintenance and other tasks},
and finally, functions used to, and in the process of, drawing and
retrieving data from the database fall within \textbf{extraction}.

\hypertarget{the-data}{%
\subsection{The data}\label{the-data}}

\begin{Shaded}
\begin{Highlighting}[]
\NormalTok{DB.path}\OtherTok{=}\StringTok{"/rfs/LRWE\_Proj59/jcfs2/Test"}
\NormalTok{FILE.path }\OtherTok{=} \StringTok{"/rfs/LRWE\_Proj59/jcfs2/Test"}
\end{Highlighting}
\end{Shaded}

\hypertarget{loading}{%
\subsection{Loading}\label{loading}}

Before loading any data into the database, it is best practice to
understand the layout of the \emph{raw} data. This can be achieved by
either navigating to the location where the data are stored or by
employing the \texttt{list\_files} and \texttt{list\_cprd} functions.
The \texttt{list\_files} function provides a list of all files of a
specified type in a specified location. To view all text (\texttt{.txt})
files in the data location

\begin{Shaded}
\begin{Highlighting}[]
\FunctionTok{list\_files}\NormalTok{(}\AttributeTok{file\_location =}\NormalTok{ FILE.path,}
           \AttributeTok{file\_type =} \StringTok{".txt"}\NormalTok{)}
\CommentTok{\#\textgreater{} $file\_location}
\CommentTok{\#\textgreater{} [1] "/rfs/LRWE\_Proj59/jcfs2/Test"}
\CommentTok{\#\textgreater{} }
\CommentTok{\#\textgreater{} $files}
\CommentTok{\#\textgreater{}                                                 files}
\CommentTok{\#\textgreater{} 1 /rfs/LRWE\_Proj59/jcfs2/Test/hes\_patient\_19\_253R.txt}
\end{Highlighting}
\end{Shaded}

To view all compressed/\emph{zipped} (\texttt{.zip}) files in the data
location

\begin{Shaded}
\begin{Highlighting}[]
\FunctionTok{list\_files}\NormalTok{(}\AttributeTok{file\_location =}\NormalTok{ FILE.path,}
           \AttributeTok{file\_type =} \StringTok{".zip"}\NormalTok{)}
\CommentTok{\#\textgreater{} $file\_location}
\CommentTok{\#\textgreater{} [1] "/rfs/LRWE\_Proj59/jcfs2/Test"}
\CommentTok{\#\textgreater{} }
\CommentTok{\#\textgreater{} $files}
\CommentTok{\#\textgreater{}                                                             files}
\CommentTok{\#\textgreater{} 1 /rfs/LRWE\_Proj59/jcfs2/Test/305546.ms2\_Extract\_Clinical\_001.zip}
\CommentTok{\#\textgreater{} 2 /rfs/LRWE\_Proj59/jcfs2/Test/305546.ms2\_Extract\_Clinical\_002.zip}
\CommentTok{\#\textgreater{} 3 /rfs/LRWE\_Proj59/jcfs2/Test/305546.ms2\_Extract\_Clinical\_003.zip}
\CommentTok{\#\textgreater{} 4 /rfs/LRWE\_Proj59/jcfs2/Test/305546.ms2\_Extract\_Clinical\_004.zip}
\CommentTok{\#\textgreater{} 5 /rfs/LRWE\_Proj59/jcfs2/Test/305720.ms1\_Extract\_Practice\_001.zip}
\CommentTok{\#\textgreater{} 6 /rfs/LRWE\_Proj59/jcfs2/Test/305720.ms2\_Extract\_Practice\_001.zip}
\end{Highlighting}
\end{Shaded}

And to view all files in the data location, excluding files in
sub-folders

\begin{Shaded}
\begin{Highlighting}[]
\FunctionTok{list\_files}\NormalTok{(}\AttributeTok{file\_location =}\NormalTok{ FILE.path,}
           \AttributeTok{file\_type =} \StringTok{"all"}\NormalTok{)}
\CommentTok{\#\textgreater{} $file\_location}
\CommentTok{\#\textgreater{} [1] "/rfs/LRWE\_Proj59/jcfs2/Test"}
\CommentTok{\#\textgreater{} }
\CommentTok{\#\textgreater{} $files}
\CommentTok{\#\textgreater{}                                                              files}
\CommentTok{\#\textgreater{} 1  /rfs/LRWE\_Proj59/jcfs2/Test/305546.ms2\_Extract\_Clinical\_001.zip}
\CommentTok{\#\textgreater{} 2  /rfs/LRWE\_Proj59/jcfs2/Test/305546.ms2\_Extract\_Clinical\_002.zip}
\CommentTok{\#\textgreater{} 3  /rfs/LRWE\_Proj59/jcfs2/Test/305546.ms2\_Extract\_Clinical\_003.zip}
\CommentTok{\#\textgreater{} 4  /rfs/LRWE\_Proj59/jcfs2/Test/305546.ms2\_Extract\_Clinical\_004.zip}
\CommentTok{\#\textgreater{} 5  /rfs/LRWE\_Proj59/jcfs2/Test/305720.ms1\_Extract\_Practice\_001.zip}
\CommentTok{\#\textgreater{} 6  /rfs/LRWE\_Proj59/jcfs2/Test/305720.ms2\_Extract\_Practice\_001.zip}
\CommentTok{\#\textgreater{} 7                             /rfs/LRWE\_Proj59/jcfs2/Test/CCI.xlsx}
\CommentTok{\#\textgreater{} 8                          /rfs/LRWE\_Proj59/jcfs2/Test/database.db}
\CommentTok{\#\textgreater{} 9              /rfs/LRWE\_Proj59/jcfs2/Test/hes\_patient\_19\_253R.txt}
\CommentTok{\#\textgreater{} 10                          /rfs/LRWE\_Proj59/jcfs2/Test/ons\_lt.csv}
\CommentTok{\#\textgreater{} 11                             /rfs/LRWE\_Proj59/jcfs2/Test/patient}
\end{Highlighting}
\end{Shaded}

From the above it can be seen that there are six zipped text files
(\texttt{.zip}), one excel file (\texttt{.xlsx}), one standard text file
(\texttt{.txt}) and a folder, \texttt{patient}. In order to view CPRD
GOLD specific files, corresponding to names CPRD GOLD datasets
(\emph{patient, practice, staff, consultation, clinical, additional
clinical details, referral, immunisation, test and therapy}), the
\texttt{list\_cprd} function is used. This function provides the core
functionality to the loading of CPRD GOLD data, generating a list of
CPRD GOLD specific files in the specified location.

\begin{Shaded}
\begin{Highlighting}[]
\FunctionTok{list\_cprd}\NormalTok{(}\AttributeTok{file\_location =}\NormalTok{ FILE.path,}
          \AttributeTok{folder =}\NormalTok{ F, }
          \AttributeTok{zip =}\NormalTok{ T)}
\CommentTok{\#\textgreater{} $file\_location}
\CommentTok{\#\textgreater{} [1] "/rfs/LRWE\_Proj59/jcfs2/Test"}
\CommentTok{\#\textgreater{} }
\CommentTok{\#\textgreater{} $all\_files\_tables}
\CommentTok{\#\textgreater{}                                                             files    table}
\CommentTok{\#\textgreater{} 1 /rfs/LRWE\_Proj59/jcfs2/Test/305546.ms2\_Extract\_Clinical\_001.zip Clinical}
\CommentTok{\#\textgreater{} 2 /rfs/LRWE\_Proj59/jcfs2/Test/305546.ms2\_Extract\_Clinical\_002.zip Clinical}
\CommentTok{\#\textgreater{} 3 /rfs/LRWE\_Proj59/jcfs2/Test/305546.ms2\_Extract\_Clinical\_003.zip Clinical}
\CommentTok{\#\textgreater{} 4 /rfs/LRWE\_Proj59/jcfs2/Test/305546.ms2\_Extract\_Clinical\_004.zip Clinical}
\CommentTok{\#\textgreater{} 5 /rfs/LRWE\_Proj59/jcfs2/Test/305720.ms1\_Extract\_Practice\_001.zip Practice}
\CommentTok{\#\textgreater{} 6 /rfs/LRWE\_Proj59/jcfs2/Test/305720.ms2\_Extract\_Practice\_001.zip Practice}
\CommentTok{\#\textgreater{} }
\CommentTok{\#\textgreater{} $tables}
\CommentTok{\#\textgreater{}      Table File\_Count}
\CommentTok{\#\textgreater{} 1 Clinical          4}
\CommentTok{\#\textgreater{} 2 Practice          2}
\end{Highlighting}
\end{Shaded}

Here the folders input is defined as FALSE, showing only zipped data
contained in the data location specified. When the folder input is TRUE,
the available data found in the \texttt{patient} folder is displayed.

\begin{Shaded}
\begin{Highlighting}[]
\FunctionTok{list\_cprd}\NormalTok{(}\AttributeTok{file\_location =}\NormalTok{ FILE.path,}
          \AttributeTok{folder =}\NormalTok{ T, }
          \AttributeTok{zip =}\NormalTok{ T)}
\CommentTok{\#\textgreater{} $file\_location}
\CommentTok{\#\textgreater{} [1] "/rfs/LRWE\_Proj59/jcfs2/Test"}
\CommentTok{\#\textgreater{} }
\CommentTok{\#\textgreater{} $all\_files\_tables}
\CommentTok{\#\textgreater{}                                                                    files}
\CommentTok{\#\textgreater{} 1 /rfs/LRWE\_Proj59/jcfs2/Test/patient/305508.ms1\_Extract\_Patient\_001.zip}
\CommentTok{\#\textgreater{} 2 /rfs/LRWE\_Proj59/jcfs2/Test/patient/305508.ms2\_Extract\_Patient\_001.zip}
\CommentTok{\#\textgreater{}     table}
\CommentTok{\#\textgreater{} 1 Patient}
\CommentTok{\#\textgreater{} 2 Patient}
\CommentTok{\#\textgreater{} }
\CommentTok{\#\textgreater{} $tables}
\CommentTok{\#\textgreater{}     Table File\_Count}
\CommentTok{\#\textgreater{} 1 Patient          2}
\end{Highlighting}
\end{Shaded}

In the \texttt{\$table} output, the \texttt{list\_cprd} function \#\#
Extracting

\hypertarget{maintenace-and-other-tasks}{%
\subsection{Maintenace and other
tasks}\label{maintenace-and-other-tasks}}

\hypertarget{references}{%
\subsection{\# References}\label{references}}

title: ``CPRDTools'' output: rmarkdown::html\_vignette vignette:
\textgreater{} \%\VignetteIndexEntry{CPRDTools}
\%\VignetteEngine{knitr::rmarkdown} \%\VignetteEncoding{UTF-8} ---

\begin{Shaded}
\begin{Highlighting}[]
\FunctionTok{library}\NormalTok{(CPRDTools)}
\end{Highlighting}
\end{Shaded}

\hypertarget{refs}{}
\begin{CSLReferences}{1}{0}
\leavevmode\vadjust pre{\hypertarget{ref-Herrett2015}{}}%
Herrett, Emily, Arlene M. Gallagher, Krishnan Bhaskaran, Harriet Forbes,
Rohini Mathur, Tjeerd van Staa, and Liam Smeeth. 2015. {``Data
{R}esource {P}rofile: {C}linical {P}ractice {R}esearch {D}atalink
({CPRD}).''} \emph{International {J}ournal of {E}pidemiology} 44 (3):
827--36. \url{https://doi.org/10.1093/ije/dyv098}.

\leavevmode\vadjust pre{\hypertarget{ref-Padmanabhan2017}{}}%
Padmanabhan, Shivani. 2017. {``{CPRD} {G}old {D}ata {S}pecification.''}
\url{https://cprdcw.cprd.com/_docs/CPRD_GOLD_Full_Data_Specification_v2.0.pdf}.

\leavevmode\vadjust pre{\hypertarget{ref-sqlite}{}}%
SQLite. 2017. \emph{S{QL}ite}. Charlotte, {N}orth {C}aolina: {Hipp,
{W}yrick and {C}ompany, {I}nc}. \url{https://www.sqlite.org/index.html}.

\end{CSLReferences}

\end{document}
